\section{Introduction}


\section{Basic Concept}

\subsection{DI measurements}

\subsection{Variometer and Scalar measurements}

\subsection{The Programs Layout}


\section{Input and Output Formats}

\subsection{Supported input types}

Any data loaded into the MagPy analysis software is transferred into an internal data structure. This structure represents a time series with a number of keywords which represent individual columns of the original data. The following keywords (table) are generally available for all input data, although most input data only use a few of them.

KEYLIST = ['time','x','y','z','f','t1','t2','var1','var2','var3','var4','var5','dx','dy','dz','df','str1','str2','str3','str4','flag','comment','typ','sectime']


\subsection{Supported output types}

\subsection{Using databases}


\section{Application}

\subsection{Basic data analysis}

\subsubsection{Reading and Plotting}

Plotting all data from a certain directory:
\begin{verbatim}
    from dev_magpy_stream import *
    st = pmRead(path_or_url=os.path.normpath('datapath\\*'))
    st.pmplot(['x','y','z'])
\end{verbatim}
The first line denotes the import line for the required package. The second line contains the read command which load the selected data into an internal memory. The data will be stored as a structured time series, as stream ``st'', for further treatment. ``datapath'' can either be an absolute path (e.g. c:$\\$data$\\$instrument$\\$*, /data/instrument/*) or a relative path (e.g. ..$\\$data$\\$instrument\\*, ../data/instrument/*). Any function is then performed on the stream ``st'' by using the following structure: st.function(options). For plotting the time series the pmplot function is used. This function requires the

Trimming data with starttime and/or endtime limits. Starttime and endtime formats:
\begin{verbatim}
    st = pmRead(path_or_url=os.path.normpath('datapath\\*'),starttime='2011-7-8',endtime=datetime(2011,7,18,13,0,0,0))
    st.pmplot(['x','y','z'])
\end{verbatim}

\subsubsection{Writing data}

\begin{verbatim}
    from dev_magpy_stream import *
    st = pmRead(...)
    st.DoUsefulThings(options)
    st.pmwrite('outputpath')
\end{verbatim}

\subsubsection{Filtering/Smoothing/Interpolation}

The first example shows how to filter data using gaussian and linear filters.
\begin{verbatim}
    st = pmRead(...)
    st = st.filtered(filter_type='gauss',filter_width=timedelta(minutes=1))
    st.pmplot(['x','y','z'])
    st.filtered(filter_type='linear',filter_width=timedelta(minutes=60),filter_offset=timedelta(minutes=30))
    st.pmplot(['x','y','z'])
\end{verbatim}

Smoothing:
\begin{verbatim}
    st = pmRead(path_or_url=os.path.normpath('datapath\\*'),starttime='2011-7-8',endtime='2011-7-9')
    st = st.smooth(['x'],window_len=21)
    st.pmplot(['x','y','z'])
\end{verbatim}

Approximating functions and fitting:
\begin{verbatim}
    st = pmRead(path_or_url=os.path.normpath('datapath\\*'),starttime='2011-7-8',endtime='2011-7-9')
    func = st.fit(['f'],fitfunc='spline',knotstep=0.05)
    st.pmplot(['f'],function=func)
\end{verbatim}

\begin{verbatim}
    st = pmRead(path_or_url=os.path.normpath('datapath\\*'),starttime='2011-7-8',endtime='2011-7-9')
    func = st.interpol(['x','y','z'])
    st.pmplot(['f'],function=func)
\end{verbatim}

\subsubsection{Flagging data}

\begin{verbatim}
st = pmRead(path_or_url=os.path.normpath('datapath\\*'),starttime='2011-7-8',endtime='2011-7-9')
st = st.routlier()
st = st.flag_stream('f',3,"Moaing",datetime(2010,7,18,12,0,0,0),datetime(2010,7,18,13,0,0,0))
st = st.remove_flagged()
st.pmplot(['f'],function=func)
st = st.get_gaps(gapvariable=True)
st.pmplot(['f','var2'])
\end{verbatim}


\subsection{DI and Baselines}

\subsubsection{DI analysis}

\begin{verbatim}
abso = analyzeAbsFiles(path_or_url=source, alpha=3.25, beta=0.0, variopath=os.path.normpath('..\\dat\\lemi025\\*'), scalarpath=os.path.normpath('..\\dat\\didd\\*'), archivepath=os.path.normpath('..\\dat\\absolutes\\analyzed'))
abso.pmwrite('..\\dat\\output\\absolutes\\',coverage='all',mode='replace',filenamebegins='absolutes_lemi')
abso.pmplot(['x','y','z'])
\end{verbatim}
The analyzeAbsFiles function reads DI files located in directory source (e.g. os.path.normpath('..\\dat\\absolutes\\raw')). For calculation, the variometer data in variopath is used and if no scalar values are provided along with the DI file, scalar data from scalarpath is used. If the analysis is successful, DI files are moved to the archive directory. Alpha and beta describe the rotation angles. Alpha is the horizontal angle. In case of an HDZ oriented instrument, alpha corresponds to D at the time of instrument orientation. Beta describes deviations from horizontal configurations and is usually zero in case of suspended systems.



\subsubsection{Baseline calculation}

\subsubsection{Baseline stability}

\subsection{Auxiliary data}


\subsection{Advanced data analysis}

\subsubsection{Merging and Comparing records}

\subsubsection{Spectral analysis}

\subsubsection{Variation anaylsis/Strom recognition}

\begin{verbatim}
    st = pmRead(...)
    func = st.fit(['x','y','z'],fitfunc='spline',knotstep=0.1)
    st = st.aic_calc('x',timerange=timedelta(hours=1))
    st.pmplot(['x','y','z','var2'],function=func)
    fmi = st.k_fmi(fitdegree=2)
    fmi = st.k_fmi(fitfunc='spline',knotstep=0.4)
    col = st._get_column('var2')
    st = st._put_column(col,'y')
    st = st.differentiate()
    st.pmplot(['x','var2','dy','t2'],symbollist = ['-','-','-','z'])
\end{verbatim}


\subsection{Automizing analysis procedures}

\subsubsection{Scripting}

\subsubsection{Logging and Notification}

use the python logging package

\subsection{Summary of supported routines}

Show table containing all functions, their purpose and the available keys.

\section{Advanced}

\subsection{Creating import/export filters}

\subsection{Adding Analysis Functions}

\section{Discussion}

